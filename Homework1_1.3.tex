\documentclass{article}
\usepackage{amsmath,graphicx}
\usepackage{cite}
\usepackage{amsthm,amssymb,amsfonts}
\usepackage{textcomp}
\usepackage{bm}
\usepackage{algorithm}    
\usepackage{algorithmic}
\usepackage{booktabs}

\begin{document}

\title{Machine Learning, Spring 2018\\Homework 1}
\date{Due on 23:59 Mar 15, 2018\\Send to $cs282\_01@163.com$ \\with subject "Chinese name+student number+HW1"}
\maketitle

%\benu
%
%\item

\section{Preliminaries}
The weight update rule $\mathbf{w}(t+1)=\mathbf{w}(t) + y(t)\mathbf{x}(t)$ has the nice interpretation that it move in the direction of classifying $\mathbf{x}(t) $ correctly.

\paragraph{a)}Show that $y(t)\mathbf{w}^T(t)\mathbf{x}(t)<0$.[\textit{Hint}: $\mathbf{x}(t)$ is misclassified by $\mathbf{w}(t)$.]


\paragraph{b)}Show that $y(t)\mathbf{w}^T(t+1)\mathbf{x}(t)>y(t)\mathbf{w}^T(t)\mathbf{x}(t)$.[\textit{Hint}: Use update rule.]

\paragraph{c)}As far as classifying $\mathbf{x}(t)$ is concerned, argue that the move from $\mathbf{w}(t)$ to $\mathbf{x}(t+1)$ is a move `in the right direction'.




%


%\eenu
\end{document}